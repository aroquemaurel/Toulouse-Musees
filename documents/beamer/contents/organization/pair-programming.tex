	\subsection{Le << Trinome-programming >>}			% B. ANtoine
	\begin{frame}{\large Après le \textit{pair-programming}, le \textit{trinome-programming}}
		\begin{itemize}
			\item Une personne a le clavier, les deux autres regardent…
			\item Puis on tourne !
		\end{itemize}
		\vfill
		\pause
		\begin{exampleblock}{Des avantages…}
			\begin{itemize}
				\item Code plus propre, plus compréhensible
				\item Résolution des problèmes plus rapidement
				\item Moins de bugs
				\item Meilleurs interprétations des exigences
			\end{itemize}
	\end{exampleblock}
		\pause
	\vfill
	\begin{alertblock}{Et des inconvénients}
	\begin{itemize}
		\item Développement plus lent au sein d'une exigence
		\item Pas de développement des exigences en parallèle
	\end{itemize}
	\end{alertblock}
	\end{frame}
	
